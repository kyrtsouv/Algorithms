\documentclass[a4paper,10pt]{article}

\usepackage{textalpha}
\usepackage{amsmath}

\usepackage{caption}
\usepackage{float}

\usepackage{algorithm}
\usepackage{algorithmic}

\title{1\textsuperscript{η} Εργασία Στο Μάθημα Αλγόριθμοι}

\author{Ονοματεπώνυμο: Βίκτωρ Κυρτσούδης \\ ΑΕΜ: 4143}
\date{}

\renewcommand{\labelenumi}{ (\alph{enumi})}

\begin{document}
\maketitle
\begin{flushleft}

    \section*{Άσκηση 1}
    $\sqrt{\log_9{n}}, \log{\sqrt{n}}, 2^{\frac{\log{n}}{2}}, 8^{\log{n}}, \log{(4^n)}, \log{(n^n)}, (\log{(2^n)})^2, n^{\log{n}}, 2^n, 2^{n\log{n}}, n!$

    \newpage
    \section*{Άσκηση 2}
    
    \begin{algorithm}
        \caption*{Αλγόριθμος εύρεσης του πλήθους ελάχιστων μονοπατιών από τον κόμβο $s$ στον κόμβο $t$ σε έναν γράφο $G=(V, E)$}

        \begin{algorithmic}[1]
        \STATE \textbf{getAmountOfShortestPaths($G, s, t$):}
        \newline
        \STATE Θέτω $PathsTo[s] \gets 1$
        \FOR {κάθε κόμβο $v \in V - \{s\}$}
            \STATE Θέτω $PathsTo[v] \gets 0$
        \ENDFOR

        \STATE Θέτω $LevelOf[s] \gets 0$
        \FOR {κάθε κόμβο $v \in V - \{s\}$}
        \STATE Θέτω $LevelOf[v] \gets -1$
        \ENDFOR
        
        \STATE Εισάγω στην Layer[0] τον κόμβο $s$
        \STATE Θέτω τον layer counter $i \gets 0$
        \STATE Θέτω τον paths counter $cnt \gets 0$

        \WHILE{η Layer[i] δεν είναι κενή ΚΑΙ $cnt == 0$} 
            \STATE Αρχικοποιώ την Layer[i+1] ως κενή
            \FOR {κάθε κόμβο $u \in Layer[i]$}
                \FOR {κάθε κόμβο v για τον οποίο υπάρχει ακμή $(u, v)$}
                    \IF {$LevelOf[v] == -1$}
                        \STATE Θέτω $LevelOf[v] \gets i+1$
                        \STATE Εισάγω τον κόμβο $v$ στην Layer[i+1]
                    \ENDIF
                    \IF {$LevelOf[v] > LevelOf[u]$}
                        \STATE Θέτω $PathsTo[v] \gets PathsTo[u]+PathsTo[v]$
                    \ENDIF  
                    \IF {$v == t$}
                        \STATE Θέτω $cnt \gets cnt  + PathsTo[u]$
                    \ENDIF
                \ENDFOR
            \ENDFOR
            \STATE Θέτω $i \gets i + 1$
        \ENDWHILE
        \RETURN $cnt$
        \end{algorithmic}
    \end{algorithm}

    Ο αλγόριθμος βασίζεται στον BFS ο οποίος έχει χρονική πολυπλοκότητα $O(|V| + |E|)$, αλλά όλα τα ελάχιστα μονοπάτια θα εμφανιστούν στο ίδιο Layer οπότε όταν βρω πρώτη φορά το t θα είναι το τελευταίο Layer που εξετάζω και το πλήθος των ελάχιστων μονοπατιών στο t θα είναι ίσο με το άθροισμα των ελάχιστων μονοπατιών των κόμβων του προηγούμενου Layer που συνδέονται με το t.

    \newpage
    \section*{Άσκηση 3}
    \begin{enumerate}
        \item Το συγκεκριμένο πρόβλημα μπορεί να μοντελοποιηθεί με ένα κατευθυνόμενο γράφημα όπου οι κόμβοι είναι όλες οι πιθανές καταστάσεις των τριών δοχείων, δηλαδή η ποσότητα νερού που περιέχουν. Κάθε ακμή ξεκινάει από έναν κόμβο u και καταλήγει σε έναν κόμβο v αν και μόνο αν η κατάσταση του κόμβου v προκύπτει από την κατάσταση του κόμβου u με την εφαρμογή μιας επιτρεπτής εκροής.\newline
        Ξεκινώντας από την κατάσταση (a=0,b=7,c=4) πρέπει να καταλήξουμε σε οποιαδήποτε κατάσταση με b=2 ή/και c=2
        \item Μπορούμε να χρησιμοποιήσούμε είτε τον DFS είτε τον BFS αλγόριθμο αλλά με τον BFS θα βρούμε την επίλυση με τις λιγότερες απαιτούμενες ενέργειες.
    \end{enumerate}

    \newpage
    \section*{Άσκηση 4}
    \begin{enumerate}
        \item Σε κάθε κλήση της συνάρτησης εκτυπώνονται n αριθμοί και ξανακαλείται η συνάρτηση 2 φορές για $\lfloor n/2 \rfloor$ αριθμούς. Άρα η αναδρομική σχέση είναι: 

        \begin{equation*}
            T(n) = 
            \begin{cases}
            0 & \text{if } n = 0 \\
            2T(\lfloor n/2 \rfloor) + n & \text{if } n \geq 1
            \end{cases}
        \end{equation*}
        και απο Master Theorem με $a=2, b=2, c=1$ έχουμε ότι $T(n) = \Theta(n\log_2{n})$ αριθμοί εκτυπώνονται.
        \item Ο αριθμός 1 εμφανίζεται μία φορά σε κάθε κλήση της συνάρτησης άρα η αναδρομική σχέση γίνεται:
        
        \begin{equation*}
            T(n) = 
            \begin{cases}
            0 & \text{if } n = 0 \\
            2T(\lfloor n/2 \rfloor) + 1 & \text{if } n \geq 1
            \end{cases}
        \end{equation*}
        και απο Master Theorem με $a=2, b=2, c=0$ έχουμε ότι $T(n) = \Theta(n^{\log_2{2}}) = \Theta(n)$ φορές εκτυπώνεται το 1.
    \end{enumerate}

    \newpage
    \section*{Άσκηση 5}
    \begin{enumerate}
        \item Εφόσων το x είναι πλειοψηφικό στοιχείο του Α ισχύει ότι
        \begin{equation}
            \text{πλήθος } x > n/2
        \end{equation}
        Άρα υπάρχει τουλάχιστον ένα ζεύγος με δύο φορές το x και άρα θα περάσει μία φορά στο Β \textbf{(2)}.
        Έστω ότι έχουμε πίνακα Μ που έχει σε όλα τα ζεύγη τουλάχιστον μία φορά το x (λόγω του \textbf{(1)}). Αυτό σημαίνει ότι ο $B_M$ θα πάρει σίγουρα ένα x (λόγω \textbf{(2)}) και απ' όλα τα άλλα ζευγάρια θα πάρει ή ένα x, αν και τα δύο στοιχεία είναι x, ή τίποτα, αν το δεύτερο στοιχείο είναι διαφορετικό από x. Άρα ο $B_M$ θα πάρει μόνο x και άρα το x θα είναι πλειοψηφικό στοιχείο του.

        Ξεκινώντας απ'τον Μ μπορεί να προκύψει ένας Μ΄ αν πάρουμε δύο ζεύγη του Μ: [x,y] και [x,z] με y,z οποιαδήποτε στοιχεία και αντιμεταθέσουμε το y με το δεύτερο x, δηλαδή πάρουμε τα ζεύγη [x,x] και [y,z]. Τότε στον $B_{M'}$ αντί να περάσουν το πολύ δύο x, θα περάσει σίγουρα ένα x και το πολύ ένα y(αν $y=z$). Οπότε τα y που θα περάσουν στον $B_{M'}$ θα είναι σίγουρα λιγότερα από τα x, λόγω \textbf{(2)} και άρα ο $B_{M'}$ θα έχει πλειοψηφικό στοιχείο το x.

        Οπότε οι αντιμεταθέσεις τέτοιας μορφής σε έναν πίνακα Α δεν επηρεάζουν την ύπαρξη πλειοψηφικού στοιχείου στο $B_A$.

        Αφού ο κάθε πίνακας Α μπορεί να προκύψει απ' τον Μ με τις κατάλληλες αντιμεταθέσεις, αν ο Α έχει πλειοψηφικό στοιχείο το x, τότε και ο $B_A$ έχει πλειοψηφικό στοιχείο το x.

        \item Στην περίπτωση που εξετάζουμε, το περισσευούμενο στοιχείο δεν είναι το πλειοψηφικό άρα ο πίνακας Α΄ που δεν το περιέχει έχει άρτιο μέγεθος και έχει κι αυτός πλειοψηφικό το x.
        
        Αφού ο $B_A$ είναι ίσος με τον $B_{A'}$ ο οποίος έχει πλειοψηφικό το x (από \textbf{(a)}), τότε και ο $B_A$ έχει πλειοψηφικό το x.

        \item     
        \begin{algorithm}[H]
            \caption*{Αλγόριθμος εύρεσης πλειοψηφικού στοιχείου σε έναν πίνακα Α}
    
            \begin{algorithmic}[1]
            \STATE \textbf{getMajorityItem(A):}
            \STATE
            \STATE $n \gets |A|$
            \STATE Β $\gets $ τα ζεύγη του Α χωρίς το περισσευούμενο αν υπάρχει
            \FOR {κάθε ζεύγος [x,y] του Β}
                \IF {x=y}
                    \STATE Αντικαθιστώ το [x,y] με x
                \ELSE
                    \STATE Σβήνω το [x,y]
                \ENDIF
            \ENDFOR
            \STATE
            \IF {$n>0$}
                \IF {n άρτιος}
                    \STATE $b \gets $ getMajorityItem(B)
                    \IF {b είναι πλειοψηφικό στοιχείο του Β}
                        \RETURN b
                    \ENDIF
                \ELSIF {τελευταίο στοιχείο του Α είναι πλειοψηφικό στοιχείο του A}
                    \RETURN τελευταίο στοιχείο του Α
                \ELSE 
                    \STATE $b \gets $ getMajorityItem(Β)
                    \IF {b είναι πλειοψηφικό στοιχείο του Β}
                        \RETURN b
                    \ENDIF
                \ENDIF
            \ENDIF
            \STATE
            \RETURN null
            
            
            \end{algorithmic}
        \end{algorithm}

        Στην αρχή βρίσκει το $B_A$ σε γραμμικό χρόνο και αγνοεί το περισσευούμενο στοιχείο αν υπάρχει. Αν το n είναι άρτιος τότε καλεί τον εαυτό της με τον πίνακα $B_A$, ελέγχει αν το στοιχείο που επέστρεψε είναι πλειοψηφικό σε γραμμικό χρόνο και αν είναι το επιστρέφει. Αν είναι περιττός ελέγχει αν το τελευταίο στοιχείο του Α είναι πλειοψηφικό και αν είναι το επιστρέφει αλλιώς κάνει το ίδιο με πριν. Σε κάθε άλλη περίπτωση επιστρέφει null.

        \item Η πολυπλοκότητα του αλγορίθμου μπορεί να βρεθεί απ' τον αναδρομικό τύπο:
        \begin{equation*}
            T(n) = 
            \begin{cases}
            0 & \text{if } n = 0 \\
            T(\lfloor n/2 \rfloor) + n & \text{if } n \geq 1
            \end{cases}
        \end{equation*}

        Άρα από το Master Theorem με $a=1, b=2$ και $c=1$ έχουμε ότι $T(n) = \Theta(n^1) = \Theta(n)$.
    \end{enumerate}
\end{flushleft}
\end{document}